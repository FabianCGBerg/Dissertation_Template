\tolerance=1
\emergencystretch=\maxdimen
\hyphenpenalty=10000
\hbadness=10000


\begin{refsection}

\title{Acknowledgements}

\chapterimage{Sections/Headers/Appendix_D_Header.pdf} % Chapter heading image
\chapter{Acknowledgements}

\vspace{-2cm}

\section*{Acknowledgements}\index{Acknowledgements}


I would like to thank \textbf{Lisa} for supervising me and giving me the honest feedback and guidance needed to grow as a scientist, as well as \textbf{Peter} for his insightful comments and invaluable input to the projects. I always enjoyed meetings with the three of us, where I learned more about how the university works than anywhere else. I also appreciated the freedom both of you gave me in pursuing different paths and analyses, which led to some of the most interesting conclusions in my PhD and sometimes didn’t lead anywhere at all. I would also like to thank the developmental group as a whole for taking me in. I still miss the great hallway conversations with \textbf{Hans} and \textbf{Hanneke}.

A lot of gratitude goes to the friends I made in the last few years. \textbf{Selma}, \textbf{Alex}, and \textbf{Marta}, at first it was usually the four of us occupying the second floor. Somehow this escalated to hijacking couches to create space for anywhere between fifteen and thirty people (especially if free food was involved). Strictly 12:00 was lunch time and we made sure to fetch anyone who missed the mark. \textbf{Aline}, your hugs and support were unmatched, where would I be without those? There’s also nobody I’d rather drink 30 days of cornstarch for than you. I liked it so much I did it twice! \textbf{Alix}, seeing you speed through the halls gave me energy, your perseverance and strength was inspiring. I also never felt more special than when you invited me to come see your cells. \textbf{Danny}, you were a ghost at the department and your presence was always a whole event, pretty sure we’ve spent more time in the gym together than at the office. What great conversations we had when working out though. \textbf{Giada}! You were such a happy and positive presence! Meowing away and reminding us (and yourself) that \textit{“we are not our data.”} \textbf{Hannah}, always such a pleasure seeing you drag around heavy equipment back and forth from the hospital, we shared quite a few MATLAB frustrations, but that never got us down. The penguin will always be here waiting when the others kick you off the couch for scary stories. \textbf{Olof}, your positivity was infectious, and I’ll never forget the Sinterklaas visit I got at the office when you first started. \textbf{Peppe}, the best Italian at the department! You have a great presence, and I will never say no to your pizza parties, I promise to bring my protein bars! \textbf{Vaish}, did you ever complete the puzzles I made you? I enjoyed our lunch talks about plants and annoying you with terrible Bollywood movie suggestions, we’re going to watch one of those someday! \textbf{Yawen} and \textbf{Mike}, the few people beside me to still work at the office in 2020! Loved the hallway talks with you both and doing experiments with Yawen. And also thanks to the occasional visitors for lunch, \textbf{Mathilde}, \textbf{Miriam}, \textbf{Alex}, and \textbf{Linda}, for making the whole PhD experience a positive one. 

\textbf{Eliza}, you were often an inspiration for what hard work can get you, to not give up, and keep fighting for what you want! I truly admire that in you, and I am proud to call you my friend. \textbf{Anne}, one of the people I’ve known the longest! We started university at the same time but took wildly different paths. But look at us now, still here and thriving! Wonder what we’ll be up to in another 10 years. \textbf{Miriam}, we also started university together and now we are finishing a PhD together, so many years of school and we’re finally done! I’m sure we can laugh about this whole thing in a couple of years over drinks and a pizza. And a thanks to \textbf{Nina}, \textbf{Sanne}, and unofficially \textbf{Kirsten}, with whom I shared an office and plenty of laughs.

\textbf{Shanice}, I had zero doubt about who I would ask to be my paranymph, of course it had to be you! We’ve supported each other through so many things, so having you watch my back during the defense only feels natural. I don’t see this changing anytime soon, and expect many more birthdays, dinners, and walks in our future. We are a perfect mix of opposites and complements for the best kind of friendship.

\newpage
\AddLabels

\textbf{Julian}, it didn’t take me very long to decide who my other paranymph would be either. You’ve been a great support throughout the years, always there for lunch, and a wonderful boxing partner! There will be many more movie nights and gaming sessions in our future, that’s for sure. I appreciate your friendship a lot and can’t imagine I would’ve managed this PhD the way I did without all those positive moments.

To \textbf{Marie}, one of my best friends in the world, I literally could not have managed the last few years, especially the lockdown year, without your incredible support and constant positivity. You cheered me on and motivated me to get through it all. I can always rely on your honesty, and even if we disagree you are still on my side. I appreciate you more than I can say. 

\textbf{Lena}, you’ve been a great friend these past few years. All the movies, walks, and talks helped me keep my sanity. You’ve been a permanent distraction in the best kind of way and gave me the much-needed relief from my own work every now and then.

A huge thanks to my \textbf{Mom}, \textbf{Dad}, and the rest of my family, for not just supporting me during the PhD, but everything that came before and is going to come after. I would not be here today without the lessons taught to me by my parents and siblings, and I would not have had as many relaxing days if it wasn’t for the kids. Playing games with \textbf{Tycho} and \textbf{Lyna} is always a highlight and getting pictures of \textbf{Axel}, \textbf{Alexis}, \textbf{Nejc}, and \textbf{Laura} would make my day.

It would not be a complete acknowledgement without also thanking all the friends I made that I haven’t met in person yet. The wonderful community of scientists and kind individuals online that are just as important as the people you see in person. The conversations I’ve had with brilliant individuals across the globe made the work so much easier and less lonely. Special shout-outs go to wonderful people that embody the term amazeballs such as \textbf{Jolanta}, \textbf{Brooke}, \textbf{Dagmar}, \textbf{Duncan}, \textbf{Helena}, \textbf{Vee}, \textbf{Julia}, and \textbf{Eva}.

A PhD cannot be attained in a void without losing a part of yourself. I realize now more than ever how much small encouragements, a listening ear, or a short distraction can improve your life in a big way. I am grateful for everyone who has been there for me, who made me a part of their lives and were kind enough to be a part of mine. The past four years have changed me more than just gaining a title. Thank you all.





\end{refsection}
\RemoveLabels