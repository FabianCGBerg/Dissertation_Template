\tolerance=1
\emergencystretch=\maxdimen
\hyphenpenalty=10000
\hbadness=10000


\begin{refsection}

\title{Impact of this Research}

\chapterimage{Sections/Headers/Appendix_C_Header.pdf} % Chapter heading image
\chapter{Impact of this Research}

\vspace{-2cm}

\section*{Impact of this Research}\index{Impact}
\begin{abstract}
Numbers are abundant in our current society; they are everywhere, from the price tags in shops to the excel sheets we use for our jobs. These arbitrary symbols inform us about exact quantities. There is no doubt that 57 is one more than 56, a comparison that is impossible for real-world non-symbolic representations (such as two flocks of birds). Using these numbers does not come naturally; it is a skill taught to children at an age most of us do not even remember. Learning numbers is a difficult task and takes years to master fully. This dissertation focused on the beginning of that journey: how number symbols gain their numerical meaning and how the brain represents and semantically processes symbols (e.g., Western Arabic numerals and finger gestures). We especially looked at the distinction between the smaller subitizing range (numbers 1 through 4) and the non-subitizing range (more than 4). 
\end{abstract}

\newpage
\AddLabels

The first two chapters in the current work investigate whether numerical symbols (such as '4' or '7') gain their meaning by mapping directly onto non-symbolic representations of these quantities (such as ••••). In both studies, we made a distinction between the subitizing range (1-4) and non-subitizing range (6-9), measuring the behavioral performance and Event-Related Potentials (ERP) of adults (Chapter 2) and children (Chapter 3). Both studies suggest that learning through direct mapping only occurs for the subitizing range. The non-subitizing range is eventually attained through other processes, likely algorithms tied to the symbol-to-symbol associations inherent to the symbolic number systems (such as counting or a later-is-greater principle). The third and fourth chapters look at the role finger-number gestures might have in number processing. These chapters show that, in adults, these culturally acquired finger patterns still facilitate the processing of Arabic numerals, but more so for the smaller subitizing range than the larger non-subitizing range. This difference between ranges suggests that the connection between finger patterns and the quantity they represent is stronger for the smaller range, possibly due to more intensive use during early development.

The results mentioned above contribute to the investigation of numerical development by demonstrating that direct mapping between a non-symbolic quantity and a numerical symbol can only occur for the subitizing range. These results strengthen the claims of more recent models on numerical development that emphasize the subitizing range. A strength of the current dissertation is also the inclusion of neural measures (in the form of Event-Related Potentials: ERPs) besides end-point measures such as response time and accuracy. ERPs provide additional information at a millisecond-level about effort invested in early and later semantic processing stages that precede decision making. The studies in the current dissertation are among the first to show the dynamic changes in neural N1 and P2p components during the symbol learning process (Chapter 2). Such changes might not be detectable with behavioral measures alone, as Chapter 3 demonstrates. The addition of neural measures could be a key factor in distinguishing between numerical and non-numerical processes that lead to these distance effects. Chapters 4 and 5 were also among the first studies to report on the sensitivity of P2p-ERP components to finger-number gesture canonicity differences. These results broaden the P2p to a marker for embodied numerical representation in addition to symbolic and non-symbolic numbers. The current results, and those in future studies that might use these measures, could help determine how the symbolic number system and wider numerical cognition develops in children. 

\newpage
Numerical literacy is crucial in today's world, with nearly every activity involving numbers to some extent. A good grasp of informal skills (such as counting or estimating) before school is a big advantage for formal mathematical education in school. Gaining a better understanding of how children develop these early numerical skills becomes increasingly more important. A good understanding of mathematics has further implications throughout life, being linked to job success, social mobility, and even mental and physical health costs. While the current work doesn't directly influence these societal aspects of mathematics education, it lays the groundwork in studying the factors that contribute to early symbol learning. The investigation into finger-number symbols is a good example of such factors, showing that these can be beneficial to learning the first four symbols. This phase of acquiring the first four symbols is fundamental and, if replicated, could inform new education programs on how to improve number learning. Likewise, the findings that the non-subitizing range cannot be mapped directly onto non-symbolic quantities could lead to programs that focus more on the inter-symbol associations that contribute more to learning these numbers. However, more research is needed to see which processes are involved in learning the meaning of numerals in the non-subitizing range.

The current dissertation is only a small piece of the puzzle. The findings presented here can enrich current paradigms that mainly focus on behavioral measures with relatively easy-to-detect ERP markers that can be used at any developmental stage. These initial findings pave the road towards a more comprehensive model of numerical development that can describe which numerical abilities children have at which age. Such a model provides a scaffold for researchers and teachers to improve teaching methods and prepare children for an increasingly number-heavy world.

\end{refsection}
\RemoveLabels